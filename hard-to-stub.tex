Beyond simply counting mocks, we are interested in finding out which
uses of mock objects are indispensable.  We propose a heuristic that
approximates the notion of indispensability. In Java, one can usually
write a stub by implementing a class (often anonymous) extending or
implementing the abstract type. Intuitively, we say that a mock is
indispensable if it is difficult or impossible to write a stub for
that type without modifying the type; for instance, the stubbed type might
not be instantiable (e.g. it has no accessible parameters). We believe that
the ``classic'' view would use mocks mostly for indispensable types, while
the ``mockist'' view would use mocks for types whether they are indispensable or not.
We also verify the admonition in the early mock papers about mocking only objects
that are close to the object under test.

Admittedly, our heuristics are not perfect; some characteristics contributing
to indispensability are difficult to compute automatically. The goal of our heuristics
is simply to identify some features of classes that make them difficult to stub.
There are classes which are difficult to stub for reasons that our heuristics don't capture.
That is, we designed our heuristics to have false negatives---missing some indispensable
classes---but few false positives. Any class that is identified as indispensable is,
in our opinion, easier to mock than to stub.

We built the indispensability analysis on top of \textsc{MockDetector}
using facts computed by Doop
(\texttt{Var-Type}, \texttt{ClassModifier}, \texttt{Method}, \texttt{Method-Modifier}, \texttt{Param-Annotation})
and by \textsc{MockDetector} (\texttt{isMockVar}
and \texttt{isMockInvocation}). This analysis identifies difficult-to-stub classes, and
computes aggregate statistics for each benchmark.

We next justify our choice of predicates. If any predicate evaluates to true on a type $T$ which is mocked, we 
classify $T$ as difficult to stub.

\begin{itemize}
\item $T$ is a final class: although final classes may be instantiable, we cannot add test behaviour to them
\item $T$ is an interface and has more than 5 methods: for an interface, we need to implement all methods in the stub, and more than the arbitrarily-chosen threshold of 5 becomes unwieldy
\item $T$ is a class but has no public constructor: such a class may be difficult to instantiate in a test context, e.g. from a different package
\item $T$ is a class with final method \texttt{m()} which at least one test method invokes: again, we cannot add test behaviour to \texttt{m()}
\item $T$ is a class with public constructors, but all constructors have more than 5 arguments: such classes are likely to be difficult to instantiate.
\end{itemize}


%% (N = <int>, N_A = <int>, AC = Abstract Class, C = Class, I = Interface, DtS = Difficult to Stub, ctor = Constructor, *internal)
%% For any type T, if one (or more) of the below conditions are satisfied, then we classify T as DtS:
%% as C:  T is final
%% as I: T has more than N methods
%% as C or AC: T has no public ctor
%% as C or AC: T has final method M which is (needed to be) invoked in at least one test method.
%% as C or AC: T has only “annoying” public ctor(s). 
%% public ctor PC is “annoying” if one or more of the following conditions are satisfied:
%% PC has more than N_A arguments
%% PC has an argument A and A is “not nullable” and A’s type is DtS
%% argument A is not nullable if A’s type is primitive or A is annotated by @NotNull

%% *as C or AC or I: T has method (or public ctor) M and M is (needed to be) invoked in at least one test case and it is computationally expensive to execute any meaningful implementation of M.
%% [+special case about mocking Object and primitive types (+their collections) -> convenience if not condition 6]


%% alternative for 5.b: there will be errors if we modify the program and add @Nullable annotation to A
%% Checker framework will complain about any dereference of possibly-null reference types by default. (Adding @Nullable will only add checks for any nonnull to nullable assignments (source).) 
%% So if there is a possibly-null dereference error on a constructor argument C_A (or any variable pointing to C_A), then we can not pass null to easily initialize it.
%% parsing checker outputs takes a bit of work! (is there a way to export maven output into a better format like csv or json?)

%% about 4: mockito did not support stubbing final methods before version 2. after version 2 devs should activate a plugin to be able to stub final methods (source1, source2). 
%% there is no case of stubbing a final method in our benchmarks.


%% currently we have (or can super easily get) count/percentage for:
%% 1, 2 (if we decide on N), 3, 4 (have the code but no benchmark), 5.a
